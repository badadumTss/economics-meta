\documentclass[12pt, a4paper]{article}

\title{Contrasting hate speech online at Meta}

\author{ Luca Zaninotto }

\usepackage{xcolor}
\usepackage{listings}
\usepackage{hyperref}
\usepackage{multicol}
\usepackage{pdfpages}
\usepackage{tikz}
\usepackage[a4paper,width=170mm,top=10mm,bottom=22mm,includeheadfoot]{geometry}
\usepackage[style=alphabetic]{biblatex}
\addbibresource{bibliography.bib}

\hypersetup{
 pdfauthor={Luca Zaninotto},
 pdftitle={Contrasting hate speech online at Meta},
 pdfkeywords={},
 pdfsubject={},
 pdfcreator={},
 pdflang={English}
}


\begin{document}
\maketitle
\begin{abstract}
  This paper will discuss how the company Meta leads innovation on
  contrasting hatespeech on their platforms trough new and innovative
  AI models, that can detect the change in the language used to spread
  hate in a matter of weeks instead of months.
\end{abstract}
\section{Introduction}

Meta Platforms Inc., also referred to in this paper as Meta and
formerly known as Facebook Inc., is an American transnational holding
company, which owns a technological conglomerate, based in Menlo Park,
California. It is a parent organization of Facebook, Instagram,
WhatsApp and Oculus. Being one of the most valuable companies in the
world, it is also one of the Big Tech companies in U.S. information
technology, alongside Amazon, Google, Apple and Microsoft. Since its
foundation Meta followed the objective of bringing people together and
make them able to communicate easily. In the last decade more and more
people started using it, to the point that it became the massive
corporation it is today. However, some problems have emerged in the
last decade, since the amount of data it had to handle became so
big. Hate speech (HS) is one of them: content that incites to violence
and rage, targetting minorities and people based on their
ethnicity. Consumers are aware of the issue, as evidenced by the
growing number of people opting out of the platforms for reasons
related to HS \cite{art:fortuna2018survey}. At the same time, building
systems that can contrast this phenomena is really difficult and
requires a big quantity of resources and skill. Meta on its side
started contrasting HS in 2016, starting a journey in AI that lead it
to became an AI superpower, alongside other tech gigants, such as
Google, Microsoft and Amazon. The technology has not yet showed its
limits, and indeed, starts to be more and more adopted by the people
for everyday tasks, so that more and more data can be collected to
make it better.

There is still much uncertainty about the direction of this technology
and what the innovations in this fields will bring us, as research
about the inner dynamics of AI usage is still scare.

Another aspect that might still be uncertain, is how the the use of
this technology for the battle against HS is in Meta's
interests. We'll see why this is the case.


The aim of this paper is therefore to show how contrasting the HS
online benefits the business model of the company, by adding value to
its products and customers. For doing so we'll first take a look at
what are facebook business model core princilpes, and how AI plays an
important role in the orchestration of actors. Then we'll present the
new innovation lead by a team of researchers in AI from the Meta AI
Labs and finally we'll tackle how these innovations benefits the whole
business model of the company.

\section{Data and Methodologies}
\subsection{Methodology}
For the dissertation on this topic we choose to do a comparative case
study approach.

\section{Analysis and results}
\paragraph{Context}
\paragraph{Construct}
\paragraph{Configuration}
\paragraph{Cooperation}
\paragraph{Capability}
\paragraph{Change}
\section{Conclusion}

\newpage
\printbibliography[title={Bibliography}]
\end{document}