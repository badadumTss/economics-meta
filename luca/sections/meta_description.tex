\section{Introduction}

% cosa dire: Meta � sta aienda gigante statuintense che si occupa di
% social media platform. I social media sono dei posti dove la gente
% pu� postare roba. Una conseguenza inevitabile � che la gente inizia
% ad insultare e sparlare. A causa del proprio business model �
% importante che questo fenomeno non si amplifichi troppo, altrimenti
% i clienti di Meta non avrebbero pi� fiducia nella piattaforma.

% - C'� Facebook, quale � il suo business model? come si inserisce il
% fatto di contrastare l'hate speech al suo interno?

% - c'ha diversi attori al suo interno, ma la risorsa maggiore che bf
% utilizza come revenue, alla fine sono le informazioni che un utente
% mette al suo interno

% - tutto il sistema sta quindi in piedi come un bel concerto se gli
% utenti continuano ad utilizzare la piattaforma

% - l'hate speech � uno dei motivi principali per il quale nel lungo
% periodo tutto il casetllo di carte potrebbe cadere, se gli utenti sono
% disincentivati dal restare sulla piattaofrma nessun inserzionista la
% utilizzer� per sponsorizzare la propria roba e tutto il carosello si
% ferma.

% - fb quindi � incentivata a combatterlo, e lo fa con sistemi sempre
% pi� innovativi dal punto di vista ingegneristico, dato che per la mole
% del flusso di contenuti che riceve � impossibile farlo fare ad una
% persona.

% - studieremo quindi come l'introduzione di nuove AI per il
% riconoscimetno preventivo di contenuti dannosi nel lungo periodo
% faccia bene a fb.

% - questa nuova tecnologia non cade dal cielo, � il frutto del lavoro
% combinato dei ricercatori di fb ed altre piattaforme con un problema
% simile.

% - useremo il 6c framework per analizzare qusta innovazione.
