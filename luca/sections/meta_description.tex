\section{Introduction}

The flow of information to wich we are continuously exposed has
indoubtely changed our lives in ways we never imagined. The first side
of the coin is obviously the access to informations and the bettering
of life quality regarding all sorts of things [elenco]. However, the
other side of the coin regards the spread of misinformation along with
hate speech, that spreads as fast, or even faster than \emph{positive}
information. Tools to contrast this issue have emerged in the form of
AI agents that are able to recognise this kind of language, to allow
automatic systems to block early this kind of threat. Companies with
this kind of problems however are a small number, since the technical
skills required to implement this kind of solutions on a large scale
are very high. We'll explore the case study of meta, and how it leads
innovation in this sector with its own implementation of a new kind of
AI that reduces the time needed by teir models to detect language
charateristics changes in a matter of weeks, instead of months.

\subsection{Meta}
Meta is a company born in 2004 as Facebook. Since its fundation its
goal was to connect people
